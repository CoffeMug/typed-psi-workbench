\documentclass[11pt,a4paper]{article}
\usepackage{hyperref}
\usepackage{amssymb}
\usepackage{amsmath}
\usepackage{amsthm}
\usepackage{color, colortbl}
\usepackage{graphicx}
\usepackage{tikz}
\usetikzlibrary{arrows,shadows}
\usepackage[utf8]{inputenc}
\usepackage{syntax}
\usepackage{alltt}
\usepackage{rotating}
\usepackage{stmaryrd}
\usepackage[all]{xy}
\usepackage{pdfpages}
\usepackage[normalem]{ulem}
\usepackage{float}
\usepackage{mathpartir}
\usepackage{macros}
\usepackage{smacros}
\usepackage{rmacros}
\usepackage{mathtools}
\usepackage{listings}
\usepackage{smllistings}
\usepackage{enumitem}

\newtheorem{definition}{Definition}
\newtheorem{lemma}[definition]{Lemma}
\newtheorem{theorem}[definition]{Theorem}
\newtheorem{corollary}[definition]{Corollary}
\newtheorem{proposition}[definition]{Proposition}
\theoremstyle{definition}

\title{Requirements specification for the typed pi instance with assertions}
\author{Amin Khorsandi}
\def\titlerunning{Requirements specification for the typed pi instance with assertions}
\def\authorrunning{Amin Khorsandi}
\date{\today}

\begin{document}
\maketitle

In this short report we present the formal specification of the
typed piAssertion-calculus instance -- the pi instance with non-trivial assertions -- and its type
system requisites. The pi-calculus with non-trivial assertions is
an attempt to study the effect of assertions in the type system of Psi-calculi.
First we give the definition of the nominal data-types for the instance and later
we introduce the instance specific functions required by the type-checker module.

\section{Nominal data-types}

Before giving the definition of nominal data-types in piAssertion instance,
we define a set of Input/Output tags~$D\in\mathbf{D}$~used in
the assertion data-type, defined as follows:
\index{instace!piAssertion-instance}
\[
\begin{array}{rcl}
\mathbf{D} & \defn & \{\mathsf{I,O}\}\\
\end{array}
\]

The set of nominal data-types of the piAssertion instance are as follows:
\index{instace!piAssertion-instance}
\[
\begin{array}{rcl}
\mathbf{T} & \defn & \N \cup \{\mathsf{In}(n) : n \in \N\} \cup \{\mathsf{Out}(m) : m \in \N\}\\
\mathbf{C} & \defn & \{M \sch N : M,N \in \mathbf{T} \}% \cup \{M = N : M,N \in \mathbf{T} \}\\ &&
\;\cup\; \{\mathsf{Asser}(n,d) : n \in \N , d \in \textbf{D}\} \cup \{ \top \} \\
\mathbf{A} & \defn & \mathcal{P}_\mathrm{fin}(\{ (n,d) : n \in \N, d \in \mathbf{D} \}) \\
\mathbf{Ty} & \defn & \{ \mathsf{Ch}(T) : T \in \mathbf{Ty} \} 
\cup \{ \mathsf{Inp}(T) : T \in \mathbf{Ty} \} \cup \{ \mathsf{Outp}(T) : T \in \mathbf{Ty} \} \cup \mathsf{Base} \\
\end{array}
\]

Note that the assertion is a set of pairs of names and tags and implementation in~\SML{} is
based on a list of pairs of names and tags.
The equivariant operators are as follows:
\[
\begin{array}{rcl}
\ftimes & \defn & \cup \\
\unit   & \defn & \emptyset \\
\vdash  & \defn & \{ \langle \Psi, \mathsf{In}(b) \sch \mathsf{Out}(b) \rangle : (b,d) \in \Psi \land d \in \mathbf{D}\} \, \cup \\
&& \{ \langle \Psi, \mathsf{Out}(b)  \sch \mathsf{In}(b) \rangle : (b,d) \in \Psi  \} \, \cup \\
&& \{ \langle \Psi, \mathsf{In}(b)   \sch \mathsf{In}(b) \rangle : (b,\mathsf{I}) \in \Psi  \} \, \cup \\
&& \{ \langle \Psi, \mathsf{Out}(b)  \sch \mathsf{Out}(b) \rangle : (b,\mathsf{O}) \in \Psi  \} \, \cup \\
&& \{ \langle \Psi, \mathsf{Asser}(n,d) \rangle : (n,d) \in \Psi \} \, \cup \\
&& \{ \langle \Psi, \top \rangle \} 
\end{array}
\]
In the definition above, we added the conditions
\[
\mathsf{In}(b)  \sch \mathsf{In}(b) \text{~and~}
\mathsf{Out}(b)  \sch \mathsf{Out}(b)
\]
to satisfy the transitivity requirement on channel equivalence operator of the
psi-calculi frame-work; this must be impossible to entail these conditions and this
is controlled by extra typing rules explained in the next section. 

\section{Typing rules}
Typing of the terms (Input and Output terms) in the piAssertion instance depends also on
the assertions. We also have a rule for typing a condition.
The compatibility relations in this instance are defined as follows:
\[
\mathsf{Outp}(T) \looparrowleft_{\mathrm{o}} T {~~and~~}\mathsf{Inp}(T) \looparrowleft_{\mathrm{i}} T
\]

The typing rules with respect to this are shown in Figure~\ref{fig:type-rules}.

\begin{figure}[htp]
\begin{mathpar}

\inferrule*[Left=\textsc{Inp}, right={$\hspace{3mm}$}]
    {E \vdash m:\mathsf{Ch}(T) \\
     (m,\mathsf{I}) \in \Psi_E}
    {E \vdash \mathsf{In}(m) : \mathsf{Inp}(T)}

\inferrule*[Left=\textsc{Outp}, right={}]
    {E \vdash m:\mathsf{Ch}(T) \\
     (m,\mathsf{O}) \in \Psi_E}
    {E \vdash \mathsf{Out}(m) : \mathsf{Outp}(T)}

\inferrule*[Left=\textsc{CondAsser}]
    {E \vdash m:\mathsf{Ch}(T)}
    {E \vdash \mathsf{Asser}(m,d)}

\inferrule*[Left=\textsc{CondEq}]
    {E \vdash M:T \qquad
     E \vdash N:U}
    {E \vdash M\sch N}
\\
\inferrule*[Left=\textsc{Ass}]
    {E \vdash a_i : \mathsf{Ch}(T) \hspace{5mm} 1 \leq i \leq n}
    {E \vdash \{(a_1,D_1),(a_2,D_2),\ldots,(a_n,D_n)\}}

\end{mathpar}
\caption{Typing rules for terms, conditions and assertions. } 
\label{fig:type-rules}
\end{figure} 
The~$\textsc{Inp}$~rule states that an input channel is well-typed (has input type) if
the name~$m$~is a channel able to carry terms of type~$T$~and also
we have the appropriate assertion~$(m,\mathsf{I})$~in the environment of the agent.
This assertion assert that~$m$~is a channel which can only be used to receive terms.
The same statements holds for the case of~$\textsc{Outp}$~rule. To make it clear,
consider the following parallel agent:
\[
\underline{\mathsf{In}(a)}(x : Base).\overline{\mathsf{Out}(b)} x \;\, | \;\, \overline{\mathsf{Out}(a)}c
\]  

The rationale behind~$\textsc{Cond}$~rule is that the
condition~$\mathsf{Asser}(m,d)$~is well-typed if the name~$m$~has a channel type
under the type environment~$E$. In this rule~$d$~is
either~$\mathsf{I}$~or~$\mathsf{O}$~and~$T$~is an arbitrary type from~$\mathbf{Ty}$. 

\subsection{Compositionality of equivalent assertions}
The definition of compositionality of equivalent assertions is as follows:
\[
\Psi \sequivalent \Psi' \Longrightarrow \Psi \ftimes \Psi'' \sequivalent \Psi' \ftimes \Psi''
\]
In the pi with assertion instance, assertions are list of name and tag pairs.
For example the following
\[
[(a,\mathsf{I}),(b,\mathsf{O}),(c,\mathsf{I})]
\]
is a valid assertion in this instance. This assertion entails the following conditions:
\[
\mathsf{Asser}(a,\mathsf{I}), \mathsf{Asser}(b,\mathsf{O}) \text{~~and~~} \mathsf{Asser}(c,\mathsf{I}).
\]
Now we show that if we compose two equivalent assertions with another
assertion the results of the composition are still equivalent.
This is straight forward, because composition operator in this instance is
the list append function. So composing a list with another list means to
simply append the latter list to the former. Since two lists carry
elements such that they both entail the same conditions, so after appending the same
list to them, they will still contain elements that entail same conditions.
Therefore the compositionality of assertion equivalence holds.
 
\section{Constraint solving rules}
The rules for solving the transition constraints of the piAssertion instance
are presented in Figure~\ref{piAssertion rules:rules}.


\begin{figure}[t!]
\[
\begin{array}{l}
\langle (\sigma, \Psi'), (\nu\ve{a})\constr{\Psi\vdash \top} \land C \rangle
\rightarrowtail
\langle (\sigma, \Psi'), C \rangle 
\\ \\
\langle (\sigma, \Psi'), (\nu\ve{a})\constr{\Psi\vdash \mathsf{Asser}(n,d)} \land C \rangle
\rightarrowtail
\langle (\sigma, \Psi'), C \rangle \\
\qquad \text{if } ((n,d)\in\Psi') \lor (n\freshin\ve{a} \land (n,d) \in \Psi)
\\ \\
\langle (\sigma, \Psi'), (\nu\ve{a})\constr{\Psi\vdash \mathsf{Asser}(n,d)} \land C \rangle
\rightarrowtail
\langle (\sigma, \Psi' \ftimes \{(n,d)\}), C \rangle \\
\qquad \text{if } n\freshin\ve{a} \land (n,d) \notin \Psi\otimes\Psi'
\\ \\
\langle (\sigma, \Psi'), (\nu\ve{a})\constr{\Psi\vdash \mathsf{Asser}(n,d)} \land C \rangle
\rightarrowtail
\langle (\sigma[n:=m], \Psi' \}), C \rangle \\
\qquad \text{if } m,n\freshin\ve{a} \land (m,d) \in \Psi\otimes\Psi'
\\ \\
\langle (\sigma, \Psi'), (\nu\ve{a})\constr{\Psi\vdash \mathsf{D}_1(a) \sch \mathsf{D}_2(a)} \land C \rangle
\rightarrowtail
\langle (\sigma, \Psi') , C \rangle \\
\qquad\text{if } a\in\ve{a}\land (a,\mathsf{D}_i)\in\Psi%\lor (a\freshin\ve{a}\land (a,d)\in\Psi')
\\ \\
\langle (\sigma, \Psi'), (\nu\ve{a})\constr{\Psi\vdash \mathsf{D}_1(a) \sch \mathsf{D}_2(b)} \land C \rangle
\rightarrowtail
\langle (\sigma[a:=b], \Psi'\}), C[a:=b] \rangle \\
\qquad \text{if } a\freshin\ve{a}  \land b\freshin \ve{a} \land
((a,\mathsf{D}_1)\in\Psi \lor (a,\mathsf{D}_2)\in\Psi)
\\ \\
\langle (\sigma, \Psi'), (\nu\ve{a})\constr{\Psi\vdash \mathsf{D}_1(a) \sch \mathsf{D}_2(b)} \land C \rangle
\rightarrowtail
\langle (\sigma[a:=b], \Psi' \ftimes \{(b,\mathsf{D}_i)\}), C[a:=b] \rangle \\
\qquad \text{if } a\freshin\ve{a}  \land b\freshin \ve{a} \land
(a,\mathsf{D}_1)\not\in\Psi \land (a,\mathsf{D}_2)\not\in\Psi
\\ \\
\langle (\sigma, \Psi'), (\nu\ve{a})\constr{\Psi\vdash \mathsf{D}_1(a) \sch b} \land C \rangle
\rightarrowtail
\langle (\sigma[b:=\mathsf{D}_2(a)], \Psi') , C[b:=\mathsf{D}_2(a)] \rangle \\
\qquad\text{if } a \freshin \ve{a} \land b \freshin \ve{b} \land
((a,\mathsf{D}_1)\in\Psi \lor (a,\mathsf{D}_2)\in\Psi)
\\ \\
\langle (\sigma, \Psi'), (\nu\ve{a})\constr{\Psi\vdash \mathsf{D}_1(a) \sch b} \land C \rangle
\rightarrowtail
\langle (\sigma[b:=\mathsf{D}_2(a)], \Psi' \ftimes \{(a,\mathsf{D}_i)\}) , C[b:=\mathsf{D}_2(a)] \rangle \\
\qquad\text{if } a \freshin \ve{a} \land b \freshin \ve{b} \land
(a,\mathsf{D}_1)\not\in\Psi \land (a,\mathsf{D}_2)\not\in\Psi
\\ \\
\langle (\sigma, \Psi'), (\nu\ve{a})\constr{\Psi\vdash a \sch b} \land C \rangle
\rightarrowtail\\\qquad\qquad
\langle (\sigma[a:=\mathsf{D}_1(c)][b:=\mathsf{D}_2(c)], \Psi') , 
(\nu\ve{a})\constr{\Psi\vdash \mathsf{D}_1(c)  \sch \mathsf{D}_2(c)} \land C[a:=\mathsf{D}_1(c)][b:=\mathsf{D}_2(c)] \rangle \\
\qquad\text{if }  a,b\freshin\ve{a} \land c \text{~fresh}
\\ \\
\langle (\sigma, \Psi'), (\nu\ve{a})\constr{\Psi\vdash \mathsf{In}(a) \sch \mathsf{Out}(b)} \land C \rangle
\rightarrowtail
\blacksquare \\
\qquad\text{if } (\Psi'' \nvdash \mathsf{Asser}(a,\mathsf{I})  \land a \in \ve{a}) \lor (\Psi'' \nvdash \mathsf{Asser}(b,\mathsf{O}) \land b \in
  \ve{b})
\\
\end{array}
\]
\caption%
{Untyped piAssertion constraint refinement transition rules. 
}
\label{piAssertion rules:rules}
\end{figure}



\end{document}
